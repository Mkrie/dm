\documentclass[a4paper,12pt]{article}

\usepackage{cmap}
\usepackage{mathtext}
\usepackage[T2A]{fontenc}
\usepackage[utf8]{inputenc}
\usepackage[english,russian]{babel}
\usepackage{indentfirst}
\frenchspacing


\usepackage{amsmath,amsfonts,amssymb,amsthm,mathtools} % AMS
\usepackage{icomma}

\DeclareMathOperator{\sgn}{\mathop{sgn}}

\newcommand*{\hm}[1]{#1\nobreak\discretionary{}
{\hbox{$\mathsurround=0pt #1$}}{}}

\usepackage{graphicx}
\graphicspath{{images/}{images2/}}
\setlength\fboxsep{3pt}
\setlength\fboxrule{1pt}
\usepackage{wrapfig}

\usepackage{array,tabularx,tabulary,booktabs}
\usepackage{longtable}
\usepackage{multirow}

\theoremstyle{plain}
\newtheorem{theorem}{Теорема}[section]
\newtheorem{proposition}[theorem]{Утверждение}
 
\theoremstyle{definition}
\newtheorem{corollary}{Следствие}[theorem]
\newtheorem{problem}{Задача}[section]
 
\theoremstyle{remark}
\newtheorem*{nonum}{Решение}

\usepackage{etoolbox}


\usepackage{extsizes}
\usepackage{geometry}
	\geometry{top=25mm}
	\geometry{bottom=35mm}
	\geometry{left=35mm}
	\geometry{right=20mm}

\usepackage{setspace}

\usepackage{lastpage}

\usepackage{soul}

\usepackage{hyperref}
\usepackage[usenames,dvipsnames,svgnames,table,rgb]{xcolor}
\hypersetup{
    unicode=true,
    pdftitle={Заголовок},
    pdfauthor={Автор},
    pdfsubject={Тема},
    pdfcreator={Создатель},
    pdfproducer={Производитель},
    pdfkeywords={keyword1} {key2} {key3},
    colorlinks=true,
    linkcolor=red,
    citecolor=black,
    filecolor=magenta,
    urlcolor=cyan
}

\usepackage{csquotes} % Еще инструменты для ссылок

%\usepackage[style=authoryear,maxcitenames=2,backend=biber,sorting=nty]{biblatex}

\usepackage{multicol} % Несколько колонок

\usepackage{tikz} % Работа с графикой
\usepackage{pgfplots}
\usepackage{pgfplotstable}

%\author{}
%\title{}
%\date{}

\begin{document} % конец преамбулы, начало документа
\subsection*{\begin{center}
Тест 1
\end{center}}
\begin{description}
\item[1.]
Из набора $\{1, 2, \ldots, 15\}$ мы выбираем $3$-элементные подмножества таким образом, чтобы сумма их элементов была чётной. Сколькими способами это можно сделать?
\item[2.]
Является ли функция $f(x)$
биекцией множества $\mathbb{R} \backslash \{ -3 \}$
на $\mathbb{R}$?
\[
f(x) = \dfrac{x-2}{x+3}
\]

\item[3.]
Для каждого $n \in \mathbb{N}$ множество
$A_n = \left\{ (x, y) \in \mathbb{R}^2 \, | \, 
|y| < nx \right\}$. Выясните, что представляют собой множества
$M_1 = \bigcap_{n=1}^{+\infty} A_n$ и 
$M_2 = \bigcup_{n=1}^{+\infty} A_n$.
\item[4.]
Мы составляем ряд из $n$ единиц и четырёх нулей. 
Сколько получится рядов, где не все четыре нуля стоят рядом?
\item[5.]
Сколько чётных четырёхзначных чисел образовано 
цифрами отличными друг от друга?
\item[6.]
Составим множество $M=\{0,0,\{0\},\{0,0\}\}$. 
Для каких элементов $x \in M$ выполняется $x \subset M$?
\item[7.]
Есть множество
$A_n = \left( -2^n, \frac{1}{n^2} \right)$.
Чем являются $A = \bigcap_{n=1}^{+\infty} A_n$
и $B = \bigcup_{n=1}^{+\infty} A_n$?
\item[8.]
Есть множество $A = \{ 1, 2, 3 \}$ и $B = \{ 1, 2 \}$.
Какие из следующих суждений верны:
\begin{description}
\item[a)] $B \in A$?
\item[b)] $B \in P(A)$?
\item[c)] $\{ \{ 1 \}, B \} \subset P(A)$?
\item[d)] $A \subset P(A)$?
\end{description}
\item[9.]
$A_n = \left\{ k \in \mathbb{N} \, | 
\, n \leq k \leq n^2 \right\}$.
Чем являются $A = \bigcap_{n=1}^{+\infty} A_n$ и
$B = \bigcup_{n=1}^{+\infty} A_n$
\item[10.]
На множестве $\mathbb{R}$ есть отношение 
$R: (x, y) \in \mathbb{R}$, если $x^3 - x = y^3 -y$:
\begin{description}
\item[a)] Проверьте, является ли $R$ эквивалентностью.
\item[b)] Определите класс эквивалентности.
\end{description}
\end{description}
\end{document} % конец документа





