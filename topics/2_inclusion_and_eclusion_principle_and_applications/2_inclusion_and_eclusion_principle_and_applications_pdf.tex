\documentclass[a4paper,12pt]{article}

\usepackage{cmap}
\usepackage{mathtext}
\usepackage[T2A]{fontenc}
\usepackage[utf8]{inputenc}
\usepackage[english,russian]{babel}
\usepackage{indentfirst}
\frenchspacing


\usepackage{amsmath,amsfonts,amssymb,amsthm,mathtools} % AMS
\usepackage{icomma}

\DeclareMathOperator{\sgn}{\mathop{sgn}}

\newcommand*{\hm}[1]{#1\nobreak\discretionary{}
{\hbox{$\mathsurround=0pt #1$}}{}}

\usepackage{graphicx}
\graphicspath{{images/}{images2/}}
\setlength\fboxsep{3pt}
\setlength\fboxrule{1pt}
\usepackage{wrapfig}

\usepackage{array,tabularx,tabulary,booktabs}
\usepackage{longtable}
\usepackage{multirow}

\theoremstyle{plain}
\newtheorem{theorem}{Теорема}[section]
\newtheorem{proposition}[theorem]{Утверждение}
 
\theoremstyle{definition}
\newtheorem{corollary}{Следствие}[theorem]
\newtheorem{problem}{Задача}[section]
 
\theoremstyle{remark}
\newtheorem*{nonum}{Решение}

\usepackage{etoolbox}


\usepackage{extsizes}
\usepackage{geometry}
	\geometry{top=25mm}
	\geometry{bottom=35mm}
	\geometry{left=35mm}
	\geometry{right=20mm}

\usepackage{setspace}

\usepackage{lastpage}

\usepackage{soul}

\usepackage{hyperref}
\usepackage[usenames,dvipsnames,svgnames,table,rgb]{xcolor}
\hypersetup{
    unicode=true,
    pdftitle={Заголовок},
    pdfauthor={Автор},
    pdfsubject={Тема},
    pdfcreator={Создатель},
    pdfproducer={Производитель},
    pdfkeywords={keyword1} {key2} {key3},
    colorlinks=true,
    linkcolor=red,
    citecolor=black,
    filecolor=magenta,
    urlcolor=cyan
}

\usepackage{csquotes} % Еще инструменты для ссылок

%\usepackage[style=authoryear,maxcitenames=2,backend=biber,sorting=nty]{biblatex}

\usepackage{multicol} % Несколько колонок

\usepackage{tikz} % Работа с графикой
\usepackage{pgfplots}
\usepackage{pgfplotstable}

%\author{}
%\title{}
%\date{}

\begin{document} % конец преамбулы, начало документа

%\subsection*{1. Базовая комбинаторика, биномиальная теорема}
\subsubsection*{Основные правила}
\textbf{Правило суммы} гласит, что если $A$ и $B$ 
-- несвязанные собы­тия, причем существует $n_i$ 
возможных исходов события $A$
и $n_2$ воз­можных исхода события $B$, то возможное число 
исходов события «$A$ или $B$» равно сумме $n_1 + n_2$. 
\\ \indent
\textbf{Правило произведения} утверждает, что если дана
последователь­ность $k$ событий с $n_1$ возможными исходами 
первого, $n_2$ -- второ­го, и т.д., вплоть до $n_k$ 
возможных исходов последнего, то общее число исходов
последовательности $k$ событий равно произведению
$n_1 \cdot n_2 \ldots \cdot n_k$.
\\ \indent
Мы выбираем $k$ элементов из множества $S$ мощности $n$. 
Если при этом порядок последовательности имеет значение, 
то мы получа­ем $(n, k)$-\textbf{размещение}, а в 
противном случае -- $(n, k)$-\textbf{сочетание}.
\textbf{Размещение с повторениями} получается в том случае, 
если в по­следовательности выбираемых элементов мы 
разрешаем появляться одинаковым, иначе мы имеем дело с 
размещением без повторе­ний. 
Аналогично определяются сочетания с повторениями и без
повторений.
\subsubsection*{Биномиальная теорема}
\textbf{Бином Ньютона} -- это формула:
\[
(a+b)^n = \sum_{k=0}^{n}
C(n, k) a^{n-k} b^k \quad \text{, где} \quad
C(n, k) = \dfrac{n!}{k! (n-k)!}.
\]
\[
C(n, k) = C(n, n - k).
\]
\indent Формула Паскаля:
\[
C(n-1, k-1) + C(n-1, k) = C(n, k), \quad 0<k<n.
\]
\begin{scriptsize}
\indent Доказательство:
\begin{equation*}
\begin{split}
C(n-1, k-1) + C(n-1, k) &= 
\dfrac{(n-1)!}{(n-k)!(k-1)!} +
\dfrac{(n-1)!}{(n-k-1)! k!} = \\
&= \dfrac{(n-1)!}{(n-k-1)!(k-1)!}
\left( \dfrac{1}{n-k} + \dfrac{1}{k} \right) \\
&= \dfrac{(n-1)!}{(n-k-1)!(k-1)!}
\left( \dfrac{n}{(n-k)k} \right) = \\
&= \dfrac{n!}{(n-k)! k!} = C(n, k)
\end{split}
\end{equation*}
\end{scriptsize}
\indent Общие количества всех $(n, k)$-размещений и $(n, k)$-сочетаний, 
как с повторениями, так и без оных даны в таблице:
\begin{center}
\begin{tabular}{|c|cc|}
\hline
& {\small Порядок существенен} & {\small Порядок не существенен} 
\\
\hline
{\small Элементы повторяются} & 
{\tiny размещения с повторениями} $n^k$ &
{\tiny сочетания с повторениями} $\dfrac{(n+k-1)!}{k!(n-1)!}$
\\
{\small Элементы не повторяются} &
{\tiny размещения без повторений}
$\dfrac{n!}{(n-k)!}$ &
{\tiny сочетания без повторении}
$\dfrac{n!}{(n-k)!k!}$ \\
\hline
\end{tabular} \\
\end{center}
\indent \textbf{Теорема о перестановках} утверждает, что существует
\[
\dfrac{n!}{n_1!n_2! \cdot \ldots \cdot n_r!}
\]
различных перестановок $n$ объектов, $n_1$ из которых относятся 
к типу $1$, $n_2$ -- к типу $2$, и т.д. вплоть до 
$n_r$ объектов типа $r$. \\ \\ \indent
\textbf{Пример.} Сколькими способами можно распределить 
$15$ студен­тов по трем учебным группам по пять студентов в каждой?
\\ \indent
\textbf{Решение.} У нас есть $15$ объектов, 
которые нужно организовать в три группы по пять. 
Это можно сделать
\[
\dfrac{15!}{5!5!5!} = 68795
\]
различными способами.
\newpage
\section*{\begin{center}
Домашнее задание
\end{center}}
\begin{center}
\textbf{Часть I}
\end{center}
\begin{description}
\item[1.] У женщины в шкафу висит шесть платьев, 
пять юбок и три блузки. 
Сколько разных нарядов она может составить из своей одежды?  
\item[2.] Сколько четырехзначных чисел, 
не превосходящих $6000$, можно составить, 
используя только нечетные цифры?
\item[3.] Пусть $S$ -- множество четырехзначных чисел, 
в чьей десятичной записи участвуют цифры: 
$О$, $1$, $2$, $3$, и $6$, причем $О$ на
первом месте, естественно, стоять не может.
\begin{description}
\item[a)] Какова мощность множества $S$?
\item[б)] Сколько чисел из $S$ в своей десятичной записи 
не имеют повторяющихся цифр?
\item[в)] Как много четных среди чисел пункта (б)?
\item[г)] Сколько чисел из пункта (б) окажутся больше, 
чем $4000$?
\end{description}
\item[4.] Комитет из $20$ членов избирает председателя и 
секретаря. Сколькими способами это можно сделать?
\item[5.] \quad
\begin{description}
\item[a)] Ресторан в своем меню предлагает пять 
различных глав­ных блюд. Каждый из компании в шесть человек 
заказы­вает свое главное блюдо.
Сколько разных заказов может получить официант?
\item[б)] Цветочница продает розы четырех разных сортов. 
Сколь­ко разных букетов можно составить из дюжины роз?
\end{description}
\item[6.] Вы покупаете пять рождественских открыток в магазине,
который может предложить четыре разных типа приглянув­шихся 
Вам открыток.
\begin{description}
\item[a)] Как много наборов из пяти открыток 
ы можете купить?
\item[б)] Сколько наборов можно составить, 
если ограничиться только двумя типами открыток из четырех, 
но купить все равно пять открыток?
\end{description}
\item[7.] Воспользуйтесь формулой Паскаля для 
доказательства равенства:
\[
C(n, k) + 2 C(n, k+1) + C(n, k+2) = C(n+2, k+2), \text{при}
\quad 0 \leq k \leq n-2
\]
\item[8.] Положив в биноме Ньютона 
$a = b = 1$, покажите, что для любого 
$n = 0, 1, 2, \ldots$ справедлива формула:
\[
\sum_{k=0}^n C(n, k) = 2^n
\]
Выведите отсюда, что в множестве $S$ из $n$ 
элементов со­держится ровно $2^n$ различных подмножеств.
(Указание: определите сначала, сколько подмножеств мощ­ности 
$k$ содержится в $S$.)
\item[9.] \quad
\begin{description}
\item[a)] Сколько разных «слов» можно получить из слова
\begin{center}
«АБРАКАДАБРА»?
\end{center}
\item[б)] Сколько из них начинаются с буквы «К»?
\item[в)] В скольких из них обе буквы «Б» стоят рядом?
\end{description}
\end{description}

\subsection*{2. Принцип включения и исключения и приложения.}

\end{document} % конец документа

