\documentclass[a4paper,12pt]{article}

\usepackage{cmap}
\usepackage{mathtext}
\usepackage[T2A]{fontenc}
\usepackage[utf8]{inputenc}
\usepackage[english,russian]{babel}
\usepackage{indentfirst}
\frenchspacing


\usepackage{amsmath,amsfonts,amssymb,amsthm,mathtools} % AMS
\usepackage{icomma}

\DeclareMathOperator{\sgn}{\mathop{sgn}}

\newcommand*{\hm}[1]{#1\nobreak\discretionary{}
{\hbox{$\mathsurround=0pt #1$}}{}}

\usepackage{graphicx}
\graphicspath{{images/}{images2/}}
\setlength\fboxsep{3pt}
\setlength\fboxrule{1pt}
\usepackage{wrapfig}

\usepackage{array,tabularx,tabulary,booktabs}
\usepackage{longtable}
\usepackage{multirow}

\theoremstyle{plain}
\newtheorem{theorem}{Теорема}[section]
\newtheorem{proposition}[theorem]{Утверждение}
 
\theoremstyle{definition}
\newtheorem{corollary}{Следствие}[theorem]
\newtheorem{problem}{Задача}[section]
 
\theoremstyle{remark}
\newtheorem*{nonum}{Решение}

\usepackage{etoolbox}


\usepackage{extsizes}
\usepackage{geometry}
	\geometry{top=25mm}
	\geometry{bottom=35mm}
	\geometry{left=35mm}
	\geometry{right=20mm}

\usepackage{setspace}

\usepackage{lastpage}

\usepackage{soul}

\usepackage{hyperref}
\usepackage[usenames,dvipsnames,svgnames,table,rgb]{xcolor}
\hypersetup{
    unicode=true,
    pdftitle={Заголовок},
    pdfauthor={Автор},
    pdfsubject={Тема},
    pdfcreator={Создатель},
    pdfproducer={Производитель},
    pdfkeywords={keyword1} {key2} {key3},
    colorlinks=true,
    linkcolor=red,
    citecolor=black,
    filecolor=magenta,
    urlcolor=cyan
}

\usepackage{csquotes} % Еще инструменты для ссылок

%\usepackage[style=authoryear,maxcitenames=2,backend=biber,sorting=nty]{biblatex}

\usepackage{multicol} % Несколько колонок

\usepackage{tikz} % Работа с графикой
\usepackage{pgfplots}
\usepackage{pgfplotstable}

%\author{}
%\title{}
%\date{}

\begin{document} % конец преамбулы, начало документа

\subsection*{2. Принцип включения и исключения и приложения.}
\indent
Пусть $A$ -- конечное множество.
Обозначим через $|A|$ число его элементов,
т.е. если множество $A$ содержит $n$ элементов,
то $|A| = n$.
\\ \indent
Пусть множества $A_1$ и $A_2$ состоят из 
конечного числа элементов.
Введём обозначение $A_{12} = A_1 \cap A_2$.
Легко убедиться, что
\[
|A_1 \cup A_2| =
|A_1|+|A_2|-|A_{12}|.
\]
Это одна из важнейших формул комбиаторики,
которую называют \textit{формулой сложения}.
С её помощью можно получить формулу для
количества элементов в объединении любого числа множеств.
\\ \indent
Например, для трёх множеств(обозначая $A_{ij} = A_i \cup A_j$),
где $i = 1,2; \, j = 2, 3; \, i \neq j, A_{123} =
A_1 \cup A_2 \cup A_3$):
\begin{equation*}
\begin{split}
|A_1 \cap A_2 \cap A_3| &= 
|A_2 \cap (A_2 \cap A_3)| = 
|A_1| + |A_2 \cup A_3| - 
|A_1 \cup (A_1 \cap A_3)| = \\
&= |A_1| + |A_2| + |A_3| - |A_{23}| -
|A_{12} \cap A_{13}|
\end{split}
\end{equation*}
Учитывая, что $A_{12} \cap A_{13} = A_{123}$
окончательно получаем
\[
|A_1 \cap A_2 \cap A_3| = 
|A_1| + |A_2| + |A_3| -
|A_{12}| - |A_{13}| - |A_{23}| + |A_{123}|.
\]
Полученные выше формулы являются частными случаями общей
формулы влючений и сключений для $n$ конечных
множеств $A_1, A_2, \ldots, A_n$:
\[
\biggl | \bigcup_{i=1}^{n}A_i \biggl | = \sum_{i} | A_i | - \sum_{i<j} | A_i \cap A_j | + \sum_{i<j<k} | A_i \cap A_j \cap A_k | - \ldots + (-1)^{n-1} | A_1 \cap A_2 \cap \ldots \cap A_n |.
\]
\newpage
\section*{\begin{center}
Домашнее задание
\end{center}}
\begin{center}
\textbf{Часть I}
\end{center}
\begin{description}
\item[1.] Студенты первого курса, 
изучающие информатику в уни­верситете, 
могут посещать и дополнительные дисципли­ны. 
В этом году $25$ из них предпочли изучать 
бухгалте­рию, $27$ выбрали бизнес, а $12$ решили 
заниматься туриз­мом. 
Кроме того, было $20$ студентов, 
слушающих курс
бухгалтерии и бизнеса, пятеро изучали бухгалтерию и
туризм, а трое -- туризм и бизнес. Известно, что никто
из студентов не отважился посещать сразу три 
дополни­тельных курса. 
Сколько студентов посещали по крайней
мере один дополнительный курс? 
Сколько из них были
увлечены только туризмом?
\item[2.]
На уроке литературы учитель решил узнать, кто
из $40$ учеников читал книги $a$, $b$ и $c$.
Результаты опроса оказались таковы:
книгу $a$ читали $25$ учащихся, книгу
$b$ -- $22$, книгу $c$ -- также $22$.
Книги $a$ или $b$ читали $33$ ученика, $a$ или 
$c$ -- $32$, $b$ или $c$ -- $31$. 
Все три книги прочли $10$ учащихся.
Сколько учеников прочли только под одной книге?
Сколько учащихся не читали ни одной из этих книг?
\end{description}

\end{document} % конец документа





