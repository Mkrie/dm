\documentclass[a4paper,12pt]{article}

\usepackage{cmap}
\usepackage{mathtext}
\usepackage[T2A]{fontenc}
\usepackage[utf8]{inputenc}
\usepackage[english,russian]{babel}
\usepackage{indentfirst}
\frenchspacing


\usepackage{amsmath,amsfonts,amssymb,amsthm,mathtools} % AMS
\usepackage{icomma}

\DeclareMathOperator{\sgn}{\mathop{sgn}}

\newcommand*{\hm}[1]{#1\nobreak\discretionary{}
{\hbox{$\mathsurround=0pt #1$}}{}}

\usepackage{graphicx}
\graphicspath{{images/}{images2/}}
\setlength\fboxsep{3pt}
\setlength\fboxrule{1pt}
\usepackage{wrapfig}

\usepackage{array,tabularx,tabulary,booktabs}
\usepackage{longtable}
\usepackage{multirow}

\theoremstyle{plain}
\newtheorem{theorem}{Теорема}[section]
\newtheorem{proposition}[theorem]{Утверждение}
 
\theoremstyle{definition}
\newtheorem{corollary}{Следствие}[theorem]
\newtheorem{problem}{Задача}[section]
 
\theoremstyle{remark}
\newtheorem*{nonum}{Решение}

\usepackage{etoolbox}


\usepackage{extsizes}
\usepackage{geometry}
	\geometry{top=25mm}
	\geometry{bottom=35mm}
	\geometry{left=35mm}
	\geometry{right=20mm}

\usepackage{setspace}

\usepackage{lastpage}

\usepackage{soul}

\usepackage{hyperref}
\usepackage[usenames,dvipsnames,svgnames,table,rgb]{xcolor}
\hypersetup{
    unicode=true,
    pdftitle={Заголовок},
    pdfauthor={Автор},
    pdfsubject={Тема},
    pdfcreator={Создатель},
    pdfproducer={Производитель},
    pdfkeywords={keyword1} {key2} {key3},
    colorlinks=true,
    linkcolor=red,
    citecolor=black,
    filecolor=magenta,
    urlcolor=cyan
}

\usepackage{csquotes} % Еще инструменты для ссылок

%\usepackage[style=authoryear,maxcitenames=2,backend=biber,sorting=nty]{biblatex}

\usepackage{multicol} % Несколько колонок

\usepackage{tikz} % Работа с графикой
\usepackage{pgfplots}
\usepackage{pgfplotstable}

\newcommand{\nl}{\\ \indent}

%\author{}
%\title{}
%\date{}

\begin{document} % конец преамбулы, начало документа

%\subsection*{2. Принцип включения и исключения и приложения.}
\indent
Пусть $A$ -- конечное множество.
Обозначим через $|A|$ число его элементов,
т.е. если множество $A$ содержит $n$ элементов,
то $|A| = n$.
\\ \indent
Пусть множества $A_1$ и $A_2$ состоят из 
конечного числа элементов.
Введём обозначение $A_{12} = A_1 \cap A_2$.
Легко убедиться, что
\[
|A_1 \cup A_2| =
|A_1|+|A_2|-|A_{12}|.
\]
Это одна из важнейших формул комбиаторики,
которую называют \textit{формулой сложения}.
С её помощью можно получить формулу для
количества элементов в объединении любого числа множеств.
\\ \indent
Например, для трёх множеств(обозначая $A_{ij} = A_i \cup A_j$),
где $i = 1,2; \, j = 2, 3; \, i \neq j, A_{123} =
A_1 \cup A_2 \cup A_3$):
\begin{equation*}
\begin{split}
|A_1 \cap A_2 \cap A_3| &= 
|A_2 \cap (A_2 \cap A_3)| = 
|A_1| + |A_2 \cup A_3| - 
|A_1 \cup (A_1 \cap A_3)| = \\
&= |A_1| + |A_2| + |A_3| - |A_{23}| -
|A_{12} \cap A_{13}|
\end{split}
\end{equation*}
Учитывая, что $A_{12} \cap A_{13} = A_{123}$
окончательно получаем
\[
|A_1 \cap A_2 \cap A_3| = 
|A_1| + |A_2| + |A_3| -
|A_{12}| - |A_{13}| - |A_{23}| + |A_{123}|.
\]
Полученные выше формулы являются частными случаями общей
формулы влючений и сключений для $n$ конечных
множеств $A_1, A_2, \ldots, A_n$:
\[
\biggl | \bigcup_{i=1}^{n}A_i \biggl | = \sum_{i} | A_i | - \sum_{i<j} | A_i \cap A_j | + \sum_{i<j<k} | A_i \cap A_j \cap A_k | - \ldots + (-1)^{n-1} | A_1 \cap A_2 \cap \ldots \cap A_n |.
\]
\newpage
\section*{\begin{center}
Домашнее задание
\end{center}}
\begin{center}
\textbf{Часть I}
\end{center}
\begin{description}
\item[1.] Студенты первого курса, 
изучающие информатику в уни­верситете, 
могут посещать и дополнительные дисципли­ны. 
В этом году $25$ из них предпочли изучать 
бухгалте­рию, $27$ выбрали бизнес, а $12$ решили 
заниматься туриз­мом. 
Кроме того, было $20$ студентов, 
слушающих курс
бухгалтерии и бизнеса, пятеро изучали бухгалтерию и
туризм, а трое -- туризм и бизнес. Известно, что никто
из студентов не отважился посещать сразу три 
дополни­тельных курса. 
Сколько студентов посещали по крайней
мере один дополнительный курс? 
Сколько из них были
увлечены только туризмом?
\item[2.]
На уроке литературы учитель решил узнать, кто
из $40$ учеников читал книги $a$, $b$ и $c$.
Результаты опроса оказались таковы:
книгу $a$ читали $25$ учащихся, книгу
$b$ -- $22$, книгу $c$ -- также $22$.
Книги $a$ или $b$ читали $33$ ученика, $a$ или 
$c$ -- $32$, $b$ или $c$ -- $31$. 
Все три книги прочли $10$ учащихся.
Сколько учеников прочли только под одной книге?
Сколько учащихся не читали ни одной из этих книг?
\end{description}
\subsection*{2. Теория множеств}
\subsubsection*{Основные понятия}
\indent
\textbf{Множество} -- это совокупность объектов, 
называемых его эле­ментами.
\\ \indent
Символом $\o$ обозначается \textbf{пустое} множество, 
а $U$ -- универсальное. 
\nl
$\mathbb{N} = \{1, 2, 3, \ldots \}$ -- 
множество \textbf{натуральных} чисел.
\nl
$\mathbb{Z}= \{0, \pm 1, \pm 2 , \pm 3 \}$ --
множество \textbf{целых} чисел.
\nl
$\mathbb{Q} = \{ \frac{p}{q} \, | \,
p \in \mathbb{Z} \wedge q \in \mathbb{N}\}$ --
множество \textbf{рациональных} чисел.
\nl
$\mathbb{R} = \{ \text{все десятичные дроби} \}$ --
множество \textbf{вещественных} чисел. 
\nl
\textbf{Подмножеством} 
множества $S$ называется множество $А$, 
все эле­менты которого принадлежат $S$. 
Этот факт обозначается так: $A \subset S$.
\nl
Два множества называются \textbf{равными} 
тогда и только тогда, когда
каждое из них является подмножеством другого.
\nl
\textbf{Объединением} множеств $А$ и $В$ называется множество
\[
A \cup B  = \{ x \, | \, x \in A \vee x \in B \}
\]
\indent
\textbf{Пересечением} множеств $А$ и $В$ называется множество
\[
A \cap B = \{ x \, | \, x \in A \wedge x \in B \}
\]
\indent
\textbf{Дополнением} множества $В$ до $А$ называется множество
\[
A \setminus B = {x \, | \, x \in A \wedge x \not\in B}
\]
\indent
\textbf{Дополнением} множества $А$ 
(до универсального множества $U$) называется множество
\[
\overline{A} =
\{ x \, | \, x \not\in A \}
\]
\indent Симметрической разностью двух множеств 
$А$ и $В$ называется множество
\[
A \triangle B = 
\{
x \, | \, (x \in A \wedge x \not\in B) \vee
(x \in B \wedge x \not\in A)
\}
\]
\indent
Из любого тождества множеств можно получить 
\textbf{двойственное},
если заменить $\cap$ на $\cup$, $\o$ на $U$ и наоборот.
\nl
\textbf{Декартовым произведением} множеств 
$А$ и $В$ является множе­ство
\[
A \times B =
\{ (a, b) \, | \, A \in A \wedge b \in B \}
\]
\indent
Элементы $A \times B$ называются \textbf{упорядоченными парами}.
\nl 
\subsubsection*{\begin{center}
Мощность множеств, счетное множество и их свойства.
\end{center}}
\indent
Множества называются  равномощными, эквивалентными, 
если между ними есть взаимно-однозначное соответствие, 
то есть такое попарное соответствие, 
когда каждому элементу одного множества сопоставляется 
один-единственный элемент другого множества и наоборот, 
при этом различным элементам одного множества 
сопоставляются различные элементы другого.
\nl
Два множества, равномощные с одним и тем же третьим множеством,
равномощны. 
Если множества $M$ и $N$ равномощны, 
то и множества всех подмножеств каждого из этих множеств 
$M$  и $N$ , также равномощны.
\nl
Под подмножеством данного множества понимается такое множество,
каждый элемент которого является элементом данного множества. 
Так множество легковых автомобилей и множество грузовых
автомобилей  будут подмножествами множества автомобилей.
\nl
Мощность множества действительных чисел, 
называют мощностью континуума. 
Наименьшей бесконечной областью является мощность множества
натуральных чисел.   .   
\nl
Часто мощности называют кардинальными числами.  
Это понятие введено немецким математиком Г. Кантором. 
Если множества обозначают символическими буквами $M$, $N$, 
то кардинальные числа обозначают через  $m$, $n$. 
Г.Кантор доказал, что множество всех подмножеств данного 
множества  $М$  имеет мощность большую, чем само множество $М$.
\nl
Множество, равномощное множеству всех натуральных чисел, называется  счетным множеством
\nl
\begin{center}
\textbf{Счётные множества, их свойства}
\end{center}
\indent
Множество - равномощное множеству всех натуральных чисел 
$(1, 2, 3, \ldots, n-1)$, 
например множество целых чисел, множество чётных чисел, 
множество рациональных чисел; все другие бесконечные множества
являются несчётными бесконечными множествами. 
Это означает, 
что все элементы счётного множества можно перенумеровать, 
то есть обозначить натуральными числами. 
Говорят, также, что счётное множество имеет мощность, 
а всякое множество, равномощное с множеством всех 
подмножеств какого-нибудь счётного множества, 
имеет мощность  или мощность континуума. 
Бесконечное множество считается счётным, 
если можно установить одно-однозначное соответствие между его
элементами и натуральными числами. 
Мощность счётного множества, например, множества простых 
чисел, меньше мощности любого бесконечного несчётного множества.
Отношение между счётным множеством и бесконечным несчётным
множеством выражается следующими теоремами:
\begin{description}
\item[1)] Мощность бесконечного множества не изменяется 
от прибавления к нему счётного множества;
\item[2)] Мощность несчётного множества не изменяется от 
удаления из него  счётного множества;
\item[3)] Любое подмножество счётного множества счётно;
\item[4)] Сумма двух счётных множеств счётна;
\item[4)] Cумма конечного и счётного множества счётна;
\item[6)] Eсли множество $А$ счётно, 
то множество всех конечных последовательностей его 
элементов также счётно;
\item[7)] Множество алгебраических чисел счётно.
\end{description}
\newpage
\section*{\begin{center}
Домашнее задание
\end{center}}
\begin{center}
\textbf{Часть I}
\end{center}
\begin{description}
\item[1.]
Определите с помощью предикатов следующий множества:
\begin{equation*}
\begin{split}
&S = \{ 2, 5, 8, 11, \ldots \}; \\
&T = \{ 1, \frac{1}{3}, \frac{1}{7}, \frac{1}{15}, \ldots \}
\end{split}
\end{equation*}
\item[2.]
Что можно сказать о непустых множествах 
$A$ и $B$, 
если имеет место равенство $A \times B = B \times A$?
\nl
Непустые множества $A$ $B$ и $C$ удовлетворяют 
соотношению $A \times B = B \times C$. 
Следует ли отсюда, что $B = C$? 
Объясните свой ответ.
\item[3.]
\emph{Показательным множеством} $P(A)$ называется множество,
элементами которого являются подмножества множества $А$.
Иначе говоря, $P(A) = \{ C \, | \, C \subset A \}$.
\begin{description}
\item[a)] Найдите $P(A)$, если $A = \{1, 2, 3 \}$.
\item[б)] Докажите, что $P(A) \cap P(B) = P(A \cap B)$
для любых множеств $A$ и $B$.
\item[в)] Покажите на примере, 
что $P(A) \cup P(B)$ не всегда совпадает с $P(A \cup B)$,
\end{description}
\end{description}
\end{document}





