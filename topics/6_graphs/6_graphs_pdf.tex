\documentclass[a4paper,12pt]{article}

\usepackage{cmap}
\usepackage{mathtext}
\usepackage[T2A]{fontenc}
\usepackage[utf8]{inputenc}
\usepackage[english,russian]{babel}
\usepackage{indentfirst}
\frenchspacing


\usepackage{amsmath,amsfonts,amssymb,amsthm,mathtools} % AMS
\usepackage{icomma}

\DeclareMathOperator{\sgn}{\mathop{sgn}}

\newcommand*{\hm}[1]{#1\nobreak\discretionary{}
{\hbox{$\mathsurround=0pt #1$}}{}}

\usepackage{graphicx}
\graphicspath{{images/}{images2/}}
\setlength\fboxsep{3pt}
\setlength\fboxrule{1pt}
\usepackage{wrapfig}

\usepackage{array,tabularx,tabulary,booktabs}
\usepackage{longtable}
\usepackage{multirow}

\theoremstyle{plain}
\newtheorem{theorem}{Теорема}[section]
\newtheorem{proposition}[theorem]{Утверждение}
 
\theoremstyle{definition}
\newtheorem{corollary}{Следствие}[theorem]
\newtheorem{problem}{Задача}[section]
 
\theoremstyle{remark}
\newtheorem*{nonum}{Решение}

\usepackage{etoolbox}


\usepackage{extsizes}
\usepackage{geometry}
	\geometry{top=25mm}
	\geometry{bottom=35mm}
	\geometry{left=35mm}
	\geometry{right=20mm}

\usepackage{setspace}

\usepackage{lastpage}

\usepackage{soul}

\usepackage{hyperref}
\usepackage[usenames,dvipsnames,svgnames,table,rgb]{xcolor}
\hypersetup{
    unicode=true,
    pdftitle={Заголовок},
    pdfauthor={Автор},
    pdfsubject={Тема},
    pdfcreator={Создатель},
    pdfproducer={Производитель},
    pdfkeywords={keyword1} {key2} {key3},
    colorlinks=true,
    linkcolor=red,
    citecolor=black,
    filecolor=magenta,
    urlcolor=cyan
}

\usepackage{csquotes} % Еще инструменты для ссылок

%\usepackage[style=authoryear,maxcitenames=2,backend=biber,sorting=nty]{biblatex}

\usepackage{multicol} % Несколько колонок

\usepackage{tikz} % Работа с графикой
\usepackage{pgfplots}
\usepackage{pgfplotstable}

\newcommand{\nl}{\\ \indent}

%\author{}
%\title{}
%\date{}

\begin{document} % конец преамбулы, начало документа

%\input{./2_inclusion_and_eclusion_principle_and_applications.tex}
\subsection*{6. Графы}
\subsubsection*{\begin{center}
Основные понятия
\end{center}}
\textbf{Граф} $G = (V, E)$ состоит из множества 
$V$ чьи элементы называют вершинами графа, 
и множества $E$ его ребер, соединяющих неко­торые пары вершин.
\nl
Вершины $u$ и $v$ графа называют \textbf{смежными}, 
если они соединены каким-то ребром $e$, про которое говорят, 
что оно инцидентно вер­шинам $u$ и $v$.
\nl
\textbf{Степенью} вершины $v$ считают число 
$\delta(v)$ ребер графа, инцидент­ных $v$.
\nl
Граф, в котором существует маршрут 
(называемый \textbf{эйлеровым}),
начинающийся и заканчивающийся в одной и той же вершине и
проходящий по каждому ребру графа ровно один раз, называется
\textbf{Эйлеровым} графом. 
Связный граф с двумя или более вершинами
является эйлеровым тогда и только тогда, 
когда каждая его верши­на имеет четную степень.
\nl
\textbf{Лемма об эстафете} утверждает, 
что сумма степеней вершин про­извольного графа 
$G = (V, E)$ равна удвоенному числу его ребер.
\nl
\textbf{Простым} принято называть граф 
$G = (V, E)$ с конечным множе­ством вершин $V$ и 
конечным множеством ребер $E$, 
в котором нет петель и кратных ребер.
\nl
Логическая матрица отношения на множестве вершин 
простого гра­фа $G$, 
которое задается его ребрами, называется 
\textbf{матрицей смеж­ности}.
\nl
\textbf{Подграфом} графа $G = (V, E)$ 
называют граф $G' = (V', E')$ в
котором $E' \subset E$ и $V' \subset V$.
\nl
\textbf{Маршрутом} длины $k$ в графе называют такую
последовательность различных вершин 
$v_0, v_1, \ldots , v_k$, в которой каждая пара соседних
вершин $v_{i-1} v_i$ соединена ребром.
\nl
\textbf{Циклом} в графе является замкнутый маршрут 
$v_0, v_1, \ldots ,v_0$ у ко­торого все вершины, 
кроме первой и последней, различны.
\nl
Граф, не содержаш;ий циклов, называют \textbf{ацикличным}.
\nl
\textbf{Связным} является тот граф, в котором каждая пара 
вершин со­единена маршрутом.
\nl
Количество компонент связности графа можно подсчитать с 
помощью \textbf{алгоритма связности}.
\nl
\textbf{Гамильтоновым} называют такой цикл в графе, 
который проходит через каждую вершину графа, 
причем только один раз. Граф, в ко­тором существует 
гамильтонов цикл, называют \textbf{гамильтоновым}.
\nl
Связный ацикличный граф $G = (V, E)$ является 
\textbf{деревом}. 
Следующие утверждения о связном графе 
$G = (V, E)$ с $n$ вершинами и $m$ ребрами эквивалентны:
\begin{description}
\item[a)] $G$ -- дерево;
\item[б)] любую пару вершин $G$ связывает единственный путь;
\item[в)] $G$ связен и $m = n - 1$
\item[г)] $G$ связен, а удаление любого его ребра 
нарушает это свойство;
\item[д)] $G$ ацикличен, 
но соединяя любую пару вершин новым ребром, мы получаем цикл.
\end{description}
\textbf{Остовным деревом} графа $G$ называют такой его 
подграф, ко­торый является деревом и содержит все вершины 
графа $G$. Алго­ритм поиска минимального остовного дерева 
позволяет най­ти остовное дерево минимального общего веса 
в нагруженном графе и может быть использован для решения 
задачи поиска кратчайшero соединения.
\nl
Дерево с одной выделенной вершиной называют 
\textbf{деревом с кор­нем}, 
а выделенную вершину -- его \textbf{корнем}. 
Вершины, стоявшие не­посредственно под вершиной 
$v$ (и соединенные с ней ребрами), 
на­зываются \textbf{сыновьями} вершины $v$. 
Вершины, расположенные в са­мом низу дерева 
(они не имеют сыновей), называются \textbf{листьями}.
Вершины, отличные от корня и листьев, 
называют \textbf{внутренними}
вершинами графа. 
\textbf{Нулевое дерево} -- 
это дерево, не имеющее ни одной вершины.
\nl
Каждая вершина дерева с корнем $Т$ является корнем 
какого-то дру­гого дерева, 
называемого \textbf{поддеревом} $Т$. 
В \textbf{двоичном дереве с корнем} каждая вершина имеет не более 
двух сыновей, а два подде­рева вершины $V$ называют \textbf{левым} 
и \textbf{правым} поддеревьями, ассо­циированными с $V$. 
Двоичное дерево с корнем называют \textbf{полным}, если каждая 
его вершина, за исключением листьев, имеет ровно по
два сына.
\nl
\textbf{Глубиной} вершины $v$ дерева с корнем 
$T$ принято считать дли­ну единственного маршрута, 
соединяющего ее с корнем. 
\textbf{Глубиной графа} $Т$ называют максимальную 
глубину его вершин.
\newpage
\subsection*{\begin{center}
Демонстрационные задачи
\end{center}}
\begin{description}
\item[1.] Известно, что дерево $T$ имеет три вершины 
степени $3$ и че­тыре вершины степени $2$. 
Остальные вершины дерева имеют
степень $1$. Сколько вершин степени $1$ есть у дерева $Т$?
(Указание: обозначьте число вершин дерева $Т$ через $n$ и при­
мените лемму об эстафете.)
\end{description}
\end{document}





