\documentclass[a4paper,12pt]{article}

\usepackage{cmap}
\usepackage{mathtext}
\usepackage[T2A]{fontenc}
\usepackage[utf8]{inputenc}
\usepackage[english,russian]{babel}
\usepackage{indentfirst}
\frenchspacing


\usepackage{amsmath,amsfonts,amssymb,amsthm,mathtools} % AMS
\usepackage{icomma}

\DeclareMathOperator{\sgn}{\mathop{sgn}}

\newcommand*{\hm}[1]{#1\nobreak\discretionary{}
{\hbox{$\mathsurround=0pt #1$}}{}}

\usepackage{graphicx}
\graphicspath{{images/}{images2/}}
\setlength\fboxsep{3pt}
\setlength\fboxrule{1pt}
\usepackage{wrapfig}

\usepackage{array,tabularx,tabulary,booktabs}
\usepackage{longtable}
\usepackage{multirow}

\theoremstyle{plain}
\newtheorem{theorem}{Теорема}[section]
\newtheorem{proposition}[theorem]{Утверждение}
 
\theoremstyle{definition}
\newtheorem{corollary}{Следствие}[theorem]
\newtheorem{problem}{Задача}[section]
 
\theoremstyle{remark}
\newtheorem*{nonum}{Решение}

\usepackage{etoolbox}


\usepackage{extsizes}
\usepackage{geometry}
	\geometry{top=25mm}
	\geometry{bottom=35mm}
	\geometry{left=35mm}
	\geometry{right=20mm}

\usepackage{setspace}

\usepackage{lastpage}

\usepackage{soul}

\usepackage{hyperref}
\usepackage[usenames,dvipsnames,svgnames,table,rgb]{xcolor}
\hypersetup{
    unicode=true,
    pdftitle={Заголовок},
    pdfauthor={Автор},
    pdfsubject={Тема},
    pdfcreator={Создатель},
    pdfproducer={Производитель},
    pdfkeywords={keyword1} {key2} {key3},
    colorlinks=true,
    linkcolor=red,
    citecolor=black,
    filecolor=magenta,
    urlcolor=cyan
}

\usepackage{csquotes} % Еще инструменты для ссылок

%\usepackage[style=authoryear,maxcitenames=2,backend=biber,sorting=nty]{biblatex}

\usepackage{multicol} % Несколько колонок

\usepackage{tikz} % Работа с графикой
\usepackage{pgfplots}
\usepackage{pgfplotstable}

%\author{}
%\title{}
%\date{}

\begin{document} % конец преамбулы, начало документа
\begin{center}
{\large \textbf{Домашняя работа по графам}}
\end{center}
\begin{description}
\item[1.]
В группе имеется девять студентов. 
Каждый из них послал по сообщению каким-то трем другим студентам. 
Возможна ли ситуация, при которой каждый студент получит 
сообщения от тех же трех студентов, 
кому он послал свои сообщения?
\item[2.] Пусть $G$ есть граф, 
построенный на вершинах $1, 2, \ldots, 15$, 
в котором вершины $i$ и $j$ смежны тогда и только тогда, 
когда их наибольший общий делитель больше единицы. 
Сколько компонент связности имеет такой граф?
\item[3.] Какое максимальное количество ребер может быть в простом
слабо связном ориентированном графе на $10$ вершинах, 
не являющимся сильно связным?
\item[4.] Какое максимальное число ребер может быть в простом 
двудольном неориентированном графе на $11$ вершинах?
\item[5.] Сколько различных остовных подграфов может иметь 
простой связный граф $G$, построенный на $m$ ребрах?
\item[6.] Лес — это граф, каждая компонента 
связности которого является деревом.\\ 
Рассмотрим лес, построенный на $41$ вершине и 
имеющий семь компонент связности. Сколько ребер в таком графе?
\item[7.] Дано дерево на семи вершинах. 
Известно, что в этом дереве по меньшей мере три вершины 
имеют степень $1$, и как минимум две вершины имеют степень $3$. 
Найдите последовательность степеней вершин этого графа.
\item[8.] Имеется кусок проволоки длиной $12$ сантиметров. 
На какое минимальное количество кусков его следует разрезать, 
чтобы из этих кусков можно было бы изготовить каркас кубика размерами 
$1 \times 1 \times 1$ при условии, 
что проволоку в процессе изготовления кубиков можно сгибать?
\item[9.] Сколько может быть совершенных паросочетаний в дереве на $n>2$ вершинах?
\item[10] Сколько существует совершенных паросочетаний в полном графе на $2n$ вершинах?
Дайте ответ для $n=10$.
\item[11.] Определите минимальный размер наибольшего по включению паросочетания в графе 
$C_{11}$, представляющем собой простой цикл, построенный на $11$ вершинах.
\item[12.] Найти количество $X$-насыщенных паросочетаний 
в полном двудольном графе $K_{n,m}$, где $X$ -- это доля меньшего размера.
Решите эту задачу для $n=8$, $m = 23$.
\item[13.] В полном графе $K_3$ существует единственный гамильтонов цикл, 
в полном графе $K_4$ -- три цикла. 
Вывести формулу для подсчета количества гамильтоновых циклов 
в произвольном полном графе $K_n$.
\end{description}
\end{document} % конец документа




