\documentclass[a4paper,12pt]{article}

\usepackage{cmap}
\usepackage{mathtext}
\usepackage[T2A]{fontenc}
\usepackage[utf8]{inputenc}
\usepackage[english,russian]{babel}
\usepackage{indentfirst}
\frenchspacing


\usepackage{amsmath,amsfonts,amssymb,amsthm,mathtools} % AMS
\usepackage{icomma}

\DeclareMathOperator{\sgn}{\mathop{sgn}}

\newcommand*{\hm}[1]{#1\nobreak\discretionary{}
{\hbox{$\mathsurround=0pt #1$}}{}}

\usepackage{graphicx}
\graphicspath{{images/}{images2/}}
\setlength\fboxsep{3pt}
\setlength\fboxrule{1pt}
\usepackage{wrapfig}

\usepackage{array,tabularx,tabulary,booktabs}
\usepackage{longtable}
\usepackage{multirow}

\theoremstyle{plain}
\newtheorem{theorem}{Теорема}[section]
\newtheorem{proposition}[theorem]{Утверждение}
 
\theoremstyle{definition}
\newtheorem{corollary}{Следствие}[theorem]
\newtheorem{problem}{Задача}[section]
 
\theoremstyle{remark}
\newtheorem*{nonum}{Решение}

\usepackage{etoolbox}


\usepackage{extsizes}
\usepackage{geometry}
	\geometry{top=25mm}
	\geometry{bottom=35mm}
	\geometry{left=35mm}
	\geometry{right=20mm}

\usepackage{setspace}

\usepackage{lastpage}

\usepackage{soul}

\usepackage{hyperref}
\usepackage[usenames,dvipsnames,svgnames,table,rgb]{xcolor}
\hypersetup{
    unicode=true,
    pdftitle={Заголовок},
    pdfauthor={Автор},
    pdfsubject={Тема},
    pdfcreator={Создатель},
    pdfproducer={Производитель},
    pdfkeywords={keyword1} {key2} {key3},
    colorlinks=true,
    linkcolor=red,
    citecolor=black,
    filecolor=magenta,
    urlcolor=cyan
}

\usepackage{csquotes} % Еще инструменты для ссылок

%\usepackage[style=authoryear,maxcitenames=2,backend=biber,sorting=nty]{biblatex}

\usepackage{multicol} % Несколько колонок

\usepackage{tikz} % Работа с графикой
\usepackage{pgfplots}
\usepackage{pgfplotstable}

\newcommand{\nl}{\\ \indent}

%\author{}
%\title{}
%\date{}

\begin{document} % конец преамбулы, начало документа
\subsection*{Правила перечислительной комбинаторики}
\begin{center}
Правило суммы и произведения
\end{center}
\begin{description}
\item[1.] 
Имеются пять видов конвертов без марок и 
четыре вида марок. 
Сколькими способами можно выбрать конверт с маркой 
для посылки письма?
\nl
{\tiny Ответ: 20.}
\item[2.] 
Сколько существует целых чисел между $0$ и $999$,  
содержащих ровно одну цифру $7$?
\nl
{\tiny Ответ: 243.}
\item[3.]
Сколькими способами можно выбрать на шахматной доске два поля, 
не лежащие на одной горизонтали или вертикали?
\nl
{\tiny Ответ: 1568.}
\item[4.]
Сколько чисел в диапазоне от $0$ до $999999$ не содержат 
двух рядом стоящих одинаковых цифр?
\nl
{\tiny Ответ: 597871.}
\end{description}
\begin{center}
Принцип включения-исключения
\end{center}
\begin{description}
\item[1.] Сколько целых чисел от 
$1$ до $100$ не делится ни на два, ни на три, ни на пять?

\end{description}

\end{document} % конец документа

